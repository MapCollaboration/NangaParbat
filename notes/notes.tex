\documentclass[10pt,a4paper]{article}
\usepackage{amsmath,amssymb,bm,makeidx,subfigure}
\usepackage[italian,english]{babel}
\usepackage[center,small]{caption}[2007/01/07]
\usepackage{fancyhdr}
\usepackage{color}

\definecolor{blu}{rgb}{0,0,1}
\definecolor{verde}{rgb}{0,1,0}
\definecolor{rosso}{rgb}{1,0,0}
\definecolor{viola}{rgb}{1,0,1}
\definecolor{arancio}{rgb}{1,0.5,0}
\definecolor{celeste}{rgb}{0,1,1}
\definecolor{rosa}{rgb}{1,0.3,0.5}

\oddsidemargin = 12pt
\topmargin = 0pt
\textwidth = 440pt
\textheight = 650pt

\makeindex

\begin{document}

\section{Structure of the observables}

Let us start from Eq.~(2.6) of Ref.~\cite{Scimemi:2017etj}, that is
the fully differential cross section for lepton-pair production in the
region in which the TMD factorisation applies, $i.e.$ $q_T \ll Q$. It
reads:
\begin{equation}\label{eq:crosssection}
\frac{d\sigma}{dQ^2 dy dq_T^2} =
\frac{4\pi\alpha^2(Q)\mathcal{P}}{3N_c Q^4} H(Q,\mu) \sum_q C_q(Q)
\int\frac{d^2\mathbf{b}}{4\pi} e^{i \mathbf{b}\cdot \mathbf{q}_T} x_1F_q(x_1,\mathbf{b};\mu,\zeta) x_2F_{\bar{q}}(x_2,\mathbf{b};\mu,\zeta)\,.
\end{equation}

Setting aside the unimportant factors, the interesting part has the
form of the integral:
\begin{equation}\label{eq:integral}
I_{ij}(x_1,x_2,q_T;\mu,\zeta)=\int\frac{d^2\mathbf{b}}{4\pi} e^{i \mathbf{b}\cdot \mathbf{q}_T} x_1F_i(x_1,\mathbf{b};\mu,\zeta) x_2F_{j}(x_2,\mathbf{b};\mu,\zeta)\,.
\end{equation}
where $F_{i(j)}$ are the evolved TMD PDFs and:
\begin{equation}
x_{1,2} = \frac{Q}{\sqrt{s}}e^{\pm y}\,,
\end{equation}
being $\sqrt{s}$ the centre-of-mass energy of the collision. In
actual fact, $F_{i(j)}$ only depend of the absolute value of
\textbf{b}, therefore eq.~(\ref{eq:integral}) can be written as:
\begin{equation}\label{eq:integral2}
I_{ij}(x_1,x_2,q_T;\mu,\zeta)=\frac12\int_0^\infty db\,b J_0(bq_T)  x_1
F_i(x_1,b;\mu,\zeta) x_2 F_{j}(x_2,b;\mu,\zeta)\,.
\end{equation}
where $J_0$ is the zero-th order Bessel function of the first kind
whose integral representation is:
\begin{equation}
J_0(x) = \frac1{2\pi}\int_0^{2\pi} d\theta e^{ix\cos(\theta)}\,.
\end{equation}
The single evolved TMD PDF at the final scales $\mu$ and $\zeta$
$F_i$, with $i=q,\bar{q}$(\footnote{Note that in
  eq.~(\ref{eq:crosssection}) the gluon TMD PDF $F_g$ is not
  involved.}), is obtained by multiplying the same TMD PDF at the
initial scales $\mu_0$ and $\zeta_0$ by a single evolution factor
$R_q(\mu_0,\zeta_0\rightarrow \mu,\zeta;b)$ (no convolutions and no
mixing), that is:
\begin{equation}
xF_i(x,b;\mu,\zeta) = R_q(\mu_0,\zeta_0\rightarrow \mu,\zeta;b)
xF_i(x,b;\mu_0,\zeta_0)\,\quad\mbox{with }i = g,q(=\bar{q})\,.
\end{equation}
so that eq.~(\ref{eq:integral2}) becomes:
\begin{equation}\label{eq:integral3}
I_{ij}(x_1,x_2,q_T;\mu,\zeta)=\frac12\int_0^\infty db\,b J_0(bq_T)
\left[R_q(\mu_0,\zeta_0\rightarrow \mu,\zeta;b)\right]^2 x_1 F_i(x_1,b;\mu_0,\zeta_0) x_2F_{j}(x_2,b;\mu_0,\zeta_0)\,.
\end{equation}

The initial scale TMD PDFs at LO in the OPE expansion, that is for
$b\ll B$ where $B$ is an unknown non-perturbative parameter that
represents the intrinsic hadron scale (see eq.~(2.27) of
Ref.~\cite{Scimemi:2017etj}), can be written as:
\begin{equation}\label{eq:LOconv}
xF_i(x,b;\mu_0,\zeta_0) = \sum_{j=g,q(\bar{q})}\int_x^1\frac{dy}{y}C_{ij}(y;\mu_0,\zeta_0)f_j\left(\frac{x}{y},\mu_0\right)\,,
\end{equation}
where $f_j$ are the collinear PDFs (including the gluon) and $C_{ij}$
are the so-called matching functions that are perturbatively
computable and are currently known to NNLO, $i.e.$
$\mathcal{O}(\alpha_s^2)$. In order account for higher orders in the
OPE and non-perturbative effects where the OPE is not valid, one
usually introduces a phenomenological non-perturbative function
$f_{\rm NP}(x,b)$ that modifies the convolution in
eq.~(\ref{eq:LOconv}) (for now we assume that this function does not
depend on the flavour of the parton, but in principle it could). The
way how $f_{\rm NP}(x,b)$ is introduced in not unique. Indeed, one can
either multiply eq.~(\ref{eq:LOconv}) by this function, that is:
\begin{equation}\label{eq:LOconvNP1}
xF_i(x,b;\mu_0,\zeta_0) \rightarrow  \sum_{j=g,q(\bar{q})}xf_{\rm NP}(x,b)\int_x^1\frac{dy}{y}C_{ij}(y;\mu_0,\zeta_0)f_j\left(\frac{x}{y},\mu_0\right)\,.
\end{equation}
or this function can be inserted in the convolution integral. This can
be done in two ways: in the first way one ``associates'' $f_{\rm NP}$ to
the collinear PDFs so that eq.~(\ref{eq:LOconv}) becomes:
\begin{equation}\label{eq:LOconvNP2}
xF_i(x,b;\mu_0,\zeta_0) \rightarrow  \sum_{j=g,q(\bar{q})}x\int_x^1\frac{dy}{y}C_{ij}(y;\mu_0,\zeta_0) \left[f_{\rm NP}\left(\frac{x}{y},b\right) f_j\left(\frac{x}{y},\mu_0\right)\right]\,,
\end{equation}
while in the second way $f_{\rm NP}$ is associated to the matching
operators, that is:
\begin{equation}\label{eq:LOconvNP3}
xF_i(x,b;\mu_0,\zeta_0) \rightarrow
\sum_{j=g,q(\bar{q})}x\int_x^1\frac{dy}{y}\left[f_{\rm
    NP}\left(y,b\right) C_{ij}(y;\mu_0,\zeta_0)\right]  f_j\left(\frac{x}{y},\mu_0\right)\,.
\end{equation}

While eqs.~(\ref{eq:LOconvNP1}) and~(\ref{eq:LOconvNP2}) appear to be
valid alternatives, eq.~(\ref{eq:LOconvNP3}) looks unnatural because
it modifies the structure of the matching coefficients that are
computable quantities (as a matter of fact this is also problematic to
implement due to the presence of plus-prescripted and
$\delta$-function terms in $C_{ij}$). Nevertheless, this seems to be
the approach taken (perhaps inadvertently) in
Ref.~\cite{Scimemi:2017etj}.

In the following we will consider the case of
eq.~(\ref{eq:LOconvNP2}). So we define:
\begin{equation}
\widetilde{f}_i\left(x,\mu_0,b\right) = xf_{\rm NP}\left(x,b\right) f_i\left(x,\mu_0\right)\,,
\end{equation}
so that:
\begin{equation}\label{eq:LOconvNPbl}
x F_i(x,b;\mu_0,\zeta_0) =
\sum_{j=g,q(\bar{q})}\int_x^1dy\,C_{ij}(y;\mu_0,\zeta_0)  \widetilde{f}_i\left(\frac{x}{y},\mu_0,b\right)\,.
\end{equation}
At this point, it is opportune to mention that, despite there seem to
be many variables, the variables $\mu_0$ and $\zeta_0$ are usually
taken to be functions of the impact parameter $b$. Therefore,
eq.~(\ref{eq:LOconvNPbl}) is a function of two variables only that we
rewrite in a simpler way as:
\begin{equation}\label{eq:LOconvNPsimp}
  x F_i(x,b) =
  \sum_{j=g,q(\bar{q})}\int_x^1dy\,C_{ij}(y,b)  \widetilde{f}_i\left(\frac{x}{y},b\right)\,.
\end{equation}

This kind of convolutions can be computed using standard interpolation
techniques by which one approximates the function $\widetilde{f}_i$ as:
\begin{equation}
  \widetilde{f}_i(x,b) = \sum_\alpha w_\alpha(x) \widetilde{f}_i(x_\alpha,b)
\end{equation}
where $x_\alpha$ is the $\alpha$-th node of an interpolation grid and
$w_\alpha$ is the interpolating function associated to that
node. Assuming for now that $x$ coincides with the $\beta$-th node of
the grid and introducing another grid in the $b$ dimension whose nodes
are indexed by $\tau$, eq.~(\ref{eq:LOconvNPsimp}) can be written as:
\begin{equation}\label{eq:LOconvNPsimp1}
  \hat{F}_{i,\beta}^\tau = \hat{C}_{ij,\beta\alpha}^\tau \hat{f}_{j,\alpha}^\tau\,.
\end{equation}
where now all sums over repeated indices are understood (except
$\tau$) and where we have used the following definitions:
\begin{equation}
\hat{F}_{i,\beta}^\tau\equiv x_\beta F_i(x_\beta,b_\tau)\,,\quad
\hat{C}_{ij,\beta\alpha}^\tau \equiv
\int_{x_\beta}^1dy\,C_{ij}(y,b_\tau)w_\alpha\left(\frac{x_\beta}{y}\right)\,,\quad
  \hat{f}_{j,\alpha}^\tau \equiv \widetilde{f}_i(x_\alpha,b_\tau)\,.
\end{equation}

Since we have to integrate over the impact parameter $b$ (see
eq.~(\ref{eq:integral2})), we need to be able to reconstruct the \
dependence of the function $F_i$ on $b$. This can be done using the
same interpolation technique. In particular, we write:
\begin{equation}
  x_\alpha F_i(x_{\alpha},b)
  x_\beta F_j(x_\beta,b) = \sum_\tau
  \widetilde{w}_{\tau}(b) \hat{F}_{i,\alpha}^\tau \hat{F}_{j,\beta}^\tau
  = \sum_\tau \widetilde{w}_{\tau}(b) \hat{C}_{ik,\alpha\gamma}^\tau
  \hat{C}_{jl,\beta\delta}^\tau \hat{f}_{k,\gamma}^\tau \hat{f}_{l,\delta}^\tau\,.
\end{equation}
Keeping in mind that $\mu_0$ and $\zeta_0$ are functions of the impact
parameter $b$, eq.~(\ref{eq:integral3}) takes the form:
\begin{equation}\label{eq:lumiInter}
I_{ij}(x_\alpha,x_\beta,q_T;\mu,\zeta)=\sum_\tau K_\tau(\mu,\zeta;q_T)\hat{C}_{ik,\alpha\gamma}^\tau
\hat{C}_{jl,\beta\delta}^\tau \hat{f}_{k,\gamma}^\tau \hat{f}_{l,\delta}^\tau\,,
\end{equation}
where we have defined:
\begin{equation}\label{eq:Kcoeff}
  K_\tau(\mu,\zeta;q_T)\equiv\frac12\int_0^\infty db\,b J_0(bq_T)
  \left[R_q(\mu,\zeta;b)\right]^2 \widetilde{w}_{\tau}(b)\,.
\end{equation}
Since the final scales $\mu$ and $\zeta$ are usually functions of the
kinematics (the typical choice is $\mu=\sqrt{\zeta}=Q$), the factors
$K_\tau$ can be precomputed for any given experiment. It should also
be noted that the function $\widetilde{w}_{\tau}(b)$ is typically
different from zero only over a finite interval in $b$, say
$\widetilde{w}_{\tau}(b)\neq 0\quad\forall\,b \in[a_\tau,b_\tau]$,
therefore the integral in eq.~(\ref{eq:Kcoeff}) reduces to:
\begin{equation}\label{eq:Kcoeff1}
  K_\tau(\mu,\zeta;q_T)\equiv\frac12\int_{a_\tau}^{b_\tau} db\,b J_0(bq_T)
  \left[R_q(\mu,\zeta;b)\right]^2 \widetilde{w}_{\tau}(b)\,,
\end{equation}
and the sum over $\tau$ in eq.~(\ref{eq:lumiInter}), that is supposed
to run over an infinite number of nodes, has to be truncated.



\begin{thebibliography}{alp}

%\cite{Scimemi:2017etj}
\bibitem{Scimemi:2017etj}
  I.~Scimemi and A.~Vladimirov,
  %``Analysis of vector boson production within TMD factorization,''
  arXiv:1706.01473 [hep-ph].
  %%CITATION = ARXIV:1706.01473;%%
  %2 citations counted in INSPIRE as of 24 Oct 2017

\end{thebibliography}

\end{document}
