\documentclass[10pt,a4paper]{article}
\usepackage{amsmath,amssymb,bm,makeidx,subfigure}
\usepackage[italian,english]{babel}
\usepackage[center,small]{caption}[2007/01/07]
\usepackage{fancyhdr}
\usepackage{color}

\definecolor{blu}{rgb}{0,0,1}
\definecolor{verde}{rgb}{0,1,0}
\definecolor{rosso}{rgb}{1,0,0}
\definecolor{viola}{rgb}{1,0,1}
\definecolor{arancio}{rgb}{1,0.5,0}
\definecolor{celeste}{rgb}{0,1,1}
\definecolor{rosa}{rgb}{1,0.3,0.5}

\oddsidemargin = 12pt
\topmargin = 0pt
\textwidth = 440pt
\textheight = 650pt

\makeindex

\begin{document}

\section{Expansion of the Sudakov form factor}

In this section I work out the expansion of the quark Sudakov form
factor to $\mathcal{O}(\alpha_s^2)$. The Sudakov form factor is
independent of the process and thus the same expansion can be used for
example for Drell-Yan and SIDIS.\footnote{Note that for
  gluon-initiated processes the expression will be different.} One
crucial point is that the expansion will be done in the so-called
$\zeta$-prescription introduced in Ref.~\cite{Scimemi:2017etj}.

The Sudakov form factor we are considering here is essentially the
square of the quark evolution factor deriving from the solution of the
Collins-Soper (CS) equations. As is well known, the CS equations
govern the evolution of TMD distributions in two independent
factorisation scales $\mu$ and $\zeta$. Usually, $\mu$ and $\zeta$ are
related to each other by assuming that $\mu=\sqrt{\zeta}$. This
particular choice reduces the degrees of freedom of the problem. On
the other hand, as observed in Ref.~\cite{Scimemi:2017etj}, this
choice does not help cure large logarithms that appear in the matching
functions. The mutual independence of $\mu$ and $\zeta$ can be
exploited to guarantee that such large logarithms are reabsorbed in
the definition of the scale $\zeta$ as a function of $\mu$ (or
viceversa). This is the basic observation at the base of the
$\zeta$-prescription that achieves this goal by assuming
$\zeta\equiv\zeta(\mu)$ and requiring that the total derivative of TMD
distribution $F$ with respect to $\mu$ is zero:
\begin{equation}\label{eq:zetapresc}
\mu^2\frac{dF\left(x,b;\mu,\zeta(\mu)\right)}{d\mu^2} = 0.
\end{equation}
This naturally leads to a differential equation in $\zeta(\mu)$
involving the CS anomalous dimensions and $\Gamma_{\rm cusp}$ that can
be solved order by order in $\alpha_s$. The exact form of the function
$\zeta(\mu)$ has to be taken into account when expanding the Sudakov
form factor.

The exact form of the Sudakov form factor, that we remind to be the
squared evolution factor deriving from the solution of the CS
equations and the evolves a pair of quark TMD distributions in $b$
space from the scales $(\mu_i,\zeta_i)$ to $(\mu_f,\zeta_f)$, is:
\begin{equation}\label{eq:sudakov}
\left[R(\mu_i,\zeta_i\rightarrow
  \mu_f,\zeta_f;b)\right]^2=\exp\left[\int_{\mu_i^2}^{\mu_f^2}\frac{d\mu^2}{\mu^2}\left(-\gamma_V+\Gamma_{\rm
    cusp}\ln\left(\frac{\mu^2}{\zeta_f}\right)\right)-2\mathcal{D}\ln\left(\frac{\zeta_f}{\zeta_i}\right)\right].
\end{equation}
where the following perturbative expansion hold:
\begin{equation}
\gamma_V = \sum_{n=1}a_s^n
\gamma_V^{(n)}\,\quad \Gamma_{\rm cusp} =
\sum_{n=1}a_s^n \Gamma_{\rm cusp}^{(n)}\,,
\end{equation}
and:
\begin{equation}
\mathcal{D} =
\sum_{n=1}a_s^{n}\sum_{k=0}^n d^{(n,k)}L^k\,,
\end{equation}
where I have defined:
\begin{equation}\label{eq:definitions}
  a_s(\mu) = \frac{\alpha_s(\mu)}{4\pi}\quad\mbox{and}\quad L\equiv \ln\left(\frac{b^2\mu_i^2}{4e^{-2\gamma_E}}\right)\,.
\end{equation}
The perturbative coefficients, $\gamma_V^{(n)}$,
$\Gamma_{\rm cusp}^{(n)}$, and $d^{(n,k)}$ are know up to the order
necessary to implement NNLL evolution.\footnote{In fact, with the only
  exception of $\Gamma_{\rm cusp}^{(4)}$, it would be possible to
  implement evolution up to N$^3$LL.} The final scales $\mu_f$ and
$\zeta_f$ are usually taken to be equal to
$\mu_f=\sqrt{\zeta}=\kappa Q$, where $Q$ is, for example the
virtuality of the exchanged photon in DIS and the invariant mass of
the lepton pair in Drell-Yan. In addition, using the
$\zeta$-prescription, we set:
\begin{equation}
\zeta_i \equiv \zeta(\mu_i) = \mu_i^2
\exp\left(\sum_{n=0}a_s^n\sum_{k=0}^{n+1}\ell^{(n,k)}L^k\right)
\end{equation}
where the coefficients $\ell^{(n,k)}$ are constants.

Before expanding the exponential we need to write its argument as a
polynomial in $\alpha_s$ (or better $a_s$) computed in $Q$. These
terms will in turn multiply powers of $\ln(Q)$. In view of the
integral over the impact parameter $b$ needed to obtain cross sections
differential in the transverse momentum $q_T$, we need to assign
$\mu_i$ a dependence on $b$. Despite this dependence is arbitrary and
needs only be dimensionally correct , the most natural choice is:
\begin{equation}
  \mu_i=\frac{C_0}{b}\,,\quad\mbox{with}\quad C_0=2e^{-\gamma_E}\,.
\end{equation}
This choice is such that $L$ defined in Eq.~(\ref{eq:definitions})
vanishes so that:
\begin{equation}
  \mathcal{D} =
  \sum_{n=1}a_s^{n}d^{(n,0)}
\end{equation}
and:
\begin{equation}
\zeta(\mu_i) =\frac{C_0^2}{b^2}
\exp\left(\sum_{n=0}a_s^n\ell^{(n,0)}\right)\,.
\end{equation}

This way the Sudakov form factor in Eq.~(\ref{eq:sudakov}) reads:
\begin{equation}\label{eq:expexp}
\begin{array}{rcl}
  \displaystyle \left[R(Q,b)\right]^2&=&\displaystyle \exp\bigg\{-\sum_{n=1}^2\int_{C_0^2/b^2}^{Q^2}\frac{d\mu^2}{\mu^2}a_s^n(\mu)\left[\gamma_V^{(n)}+\Gamma_{\rm
                                         cusp}^{(n)}\ln\left(\frac{Q^2}{\mu^2}\right)\right]\\
  \\
                                     &+&\displaystyle  2
                                         \sum_{n=1}^2a_s^{n}\left(\frac{C_0}{b}\right)d^{(n,0)}\left[-\ln\left(\frac{b^2Q^2}{C_0^2}\right)+\ell^{(0,0)}+a_s\left(\frac{C_0}{b}\right)\ell^{(1,0)}\right]\bigg\}\,,
\end{array}
\end{equation}
where we limited the contributions in the exponential to
$\mathcal{O}(a_s^2)$ that is the order we need. Now, using the RGE:
\begin{equation}
\mu^2\frac{da_s}{d\mu^2}=-\beta_0a_s^2(\mu)\,,
\end{equation}
whose solution is:
\begin{equation}
a_s(\mu) = \frac{a_s(Q)}{1+a_s(Q)\beta_0\ln(\mu^2/Q^2)}\simeq a_s(Q)\left[1+a_s(Q)\beta_0\ln(Q^2/\mu^2)+\mathcal{O}(a_s^2)\right]\,,
\end{equation}
we write every instance of $a_s$ appearing in Eq.~(\ref{eq:expexp}) in
terms of $a_s(Q)$ and finally retain only terms up to $a_s^2(Q)$:
\begin{equation}\label{eq:sudint}
\begin{array}{rcl}
  \displaystyle \left[R(Q,b)\right]^2&=&\displaystyle \exp\bigg\{-a_s(Q)\int_{C_0^2/b^2}^{Q^2}\frac{d\mu^2}{\mu^2}\left[\gamma_V^{(1)}+\Gamma_{\rm
                                         cusp}^{(1)}\ln\left(\frac{Q^2}{\mu^2}\right)\right]\\
  \\
  &-&\displaystyle a_s^2(Q) \int_{C_0^2/b^2}^{Q^2}\frac{d\mu^2}{\mu^2}\left[\gamma_V^{(2)}+\left(\Gamma_{\rm
                                         cusp}^{(2)}+\gamma_V^{(1)}\beta_0 \right)\ln\left(\frac{Q^2}{\mu^2}\right)+\Gamma_{\rm
                                         cusp}^{(1)}\beta_0\ln^2\left(\frac{Q^2}{\mu^2}\right)\right]\\
  \\
                                     &+&\displaystyle  2
                                         a_s(Q) \left[d^{(1,0)}\ell^{(0,0)}-d^{(1,0)}\ln\left(\frac{b^2Q^2}{C_0^2}\right)\right]\\
  \\ &+&\displaystyle  2 a_s^{2}(Q) \left[d^{(2,0)}\ell^{(0,0)} +
         d^{(1,0)}\ell^{(1,0)}+\left(-d^{(2,0)}+d^{(1,0)}\beta_0
         \ell^{(0,0)}\right)\ln\left(\frac{b^2Q^2}{C_0^2}\right)-d^{(1,0)}\beta_0
         \ln^2\left(\frac{b^2Q^2}{C_0^2}\right) \right]\bigg\}\,.
\end{array}
\end{equation}
The final step before carrying out the expansion is that of resolving
the integrals:
\begin{equation}
\int_{C_0^2/b^2}^{Q^2}\frac{d\mu^2}{\mu^2}\ln^k\left(\frac{Q^2}{\mu^2}\right)
= \int_{\ln(C_0^2/b^2)}^{\ln
  Q^2}d\ln\mu^2\ln^k\left(\frac{Q^2}{\mu^2}\right) = \int^{\ln(b^2Q^2/C_0^2)}_{0}dx\,x^k=\frac{1}{k+1}\ln^{k+1}\left(\frac{b^2Q^2}{C_0^2}\right)\,.
\end{equation}
If I define:
\begin{equation}
\mathcal{L}\equiv \ln\left(\frac{b^2Q^2}{C_0^2}\right)\,,
\end{equation}
the Sudakov form fact in Eq.~\ref{eq:sudint} reads:
\begin{equation}
\begin{array}{rcl}
  \displaystyle \left[R(Q,\mathcal{L})\right]^2&=&\displaystyle \exp\bigg\{a_s(Q)\left[2d^{(1,0)}\ell^{(0,0)}-\left(\gamma_V^{(1)}+2d^{(1,0)}\right)\mathcal{L}-\frac{1}{2}\Gamma_{\rm
                                         cusp}^{(1)}\mathcal{L}^2\right]\\
  \\
                                     &+&\displaystyle a_s^2(Q) \bigg[2d^{(2,0)}\ell^{(0,0)} +
                                         2d^{(1,0)}\ell^{(1,0)}+\left(-2d^{(2,0)}+2d^{(1,0)}\beta_0
                                         \ell^{(0,0)}-\gamma_V^{(2)}\right)\mathcal{L}\\
  \\
                                     &-&\displaystyle \frac12\left(4d^{(1,0)}\beta_0+\Gamma_{\rm
                                         cusp}^{(2)}+\gamma_V^{(1)}\beta_0 \right) \mathcal{L}^2-\frac13\Gamma_{\rm
                                         cusp}^{(1)}\beta_0 \mathcal{L}^3\bigg]\bigg\}\,,
\end{array}
\end{equation}
that can be conveniently written as:
\begin{equation}\label{eq:SudCompact}
  \displaystyle \left[R(Q,\mathcal{L})\right]^2=\exp\left\{\sum_{n=1}^2a_s^n(Q)\sum_{k=0}^{n+1}S^{(n,k)}\mathcal{L}^k\right\}\,,
\end{equation}
with:
\begin{equation}
\begin{array}{l}
\displaystyle S^{(1,0)} =
  2d^{(1,0)}\ell^{(0,0)}=0\,,\quad\displaystyle S^{(1,1)} =
  -\left(\gamma_V^{(1)}+2d^{(1,0)}\right)=6 C_F\,,\quad\displaystyle
                                                   S^{(1,2)} =
  -\frac{1}{2}\Gamma_{\rm cusp}^{(1)}= -2C_F\,,\\
\\
\displaystyle S^{(2,0)} = 2d^{(2,0)}\ell^{(0,0)} +
  2d^{(1,0)}\ell^{(1,0)}\,,\quad\displaystyle S^{(2,1)} =
                           \left(-2d^{(2,0)}+2d^{(1,0)}\beta_0
  \ell^{(0,0)}-\gamma_V^{(2)}\right)\,,\\
\\
\displaystyle S^{(2,2)} = -\frac12\left(4d^{(1,0)}\beta_0+\Gamma_{\rm cusp}^{(2)}+\gamma_V^{(1)}\beta_0 \right)\,,\quad\displaystyle S^{(2,3)} = -\frac13\Gamma_{\rm cusp}^{(1)}\beta_0\,.
\end{array}
\end{equation}
The values of the coefficients of the anomalous dimensions and beta
function can be read from Appendix D of
Ref.~\cite{Echevarria:2016scs}. The coefficients of the expansion of
$\zeta(\mu)$ are instead reported in Eq.~(2.29) of
Ref.~\cite{Scimemi:2017etj}. For the $\mathcal{O}(a_s)$ coefficients I
reported the explicit values. This will help check the result against
those in the literature.

Eq.~(\ref{eq:SudCompact}) can be easily expanded as up to order
$a_s^2$ as:
\begin{equation}\label{eq:exp1}
\begin{array}{rcl}
  \displaystyle \left[R(Q,\mathcal{L})\right]^2&=&\displaystyle
                                         1+a_s(Q)\sum_{k=0}^{2}S^{(1,k)}\mathcal{L}^k+a_s^2(Q)\left[\sum_{k=0}^{3}S^{(2,k)}\mathcal{L}^k+\frac12\left(\sum_{k=0}^{2}S^{(1,k)}\mathcal{L}^k\right)^2\right]+\mathcal{O}(a_s^3)\\
  \\
                                     &=&\displaystyle
                                         1+a_s(Q)\sum_{k=0}^{2}S^{(1,k)}\mathcal{L}^k+a_s^2(Q)
                                         \sum_{k=0}^{4}\widetilde{S}^{(2,k)}\mathcal{L}^k
                                         +\mathcal{O}(a_s^3)\\
\\
&\equiv&\displaystyle  1+a_s(Q)R^{(1)}+a_s^2(Q)
                                         R^{(2)}+\mathcal{O}(a_s^3)\,,
\end{array}
\end{equation}
with:
\begin{equation}
\begin{array}{l}
\displaystyle  \widetilde{S}^{(2,0)}={S}^{(2,0)}+\frac12
  \left[S^{(1,0)}\right]^2\,,\\
  \\
\displaystyle    \widetilde{S}^{(2,1)}={S}^{(2,1)}+{S}^{(1,0)}{S}^{(1,1)}\\
\\
\displaystyle    \widetilde{S}^{(2,2)}={S}^{(2,2)}+\frac12 \left[S^{(1,1)}\right]^2+{S}^{(1,2)}{S}^{(1,0)}\,,\\
\\
\displaystyle   \widetilde{S}^{(2,3)}={S}^{(2,3)}+{S}^{(1,1)}{S}^{(1,2)}\\
\\
\displaystyle    \widetilde{S}^{(2,4)}=\frac12 \left[S^{(1,2)}\right]^2\,.
\end{array}
\end{equation}

Now that the expansion of the Sudakov form factor is done to
$\mathcal{O}(a_s^2)$, it can be combined to the rest of the
perturbative quantities entering the computation of the cross
section. In the following, we will concentrate on SIDIS that involves
both TMD PDFs $F$ and TMD FFs $D$. Specifically, in $b$ space the
SIDIS cross section is a combination of terms having the following
structure:
\begin{equation}
\begin{array}{rcl}
  \displaystyle B_{ij} &=& \displaystyle
                           H(Q)\,F_i(x,b;Q)\,zD_j(z,b;Q)
                           =H(Q)\,\left[R(Q,\mathcal{L})\right]^2
                           \,F_i\left(x,b;\frac{C_0}{b}\right)\,zD_j\left(z,b;\frac{C_0}{b}\right)\\
\\
&=& \displaystyle H(Q)\,\left[R(Q,\mathcal{L})\right]^2
                           \left[\sum_{k}
    \mathcal{C}_{ik}(x,\mathcal{L}) \mathop{\otimes}_x
    f_k\left(x;Q\right)\right]\left[\sum_{l}
    \mathbb{C}_{jl}(z,\mathcal{L}) \mathop{\otimes}_z
    d_l\left(z;Q\right)\right]\\
\\
&\equiv& \displaystyle \sum_{kl}\hat{B}_{ij,kl}(x,z;b)\mathop{\otimes}_{x}f_k \mathop{\otimes}_{z}d_l \,,
\end{array}
\end{equation}
where $f_k$ and $d_l$ are the (non-perturbative) collinear PDFs and FFs,
respectively. The other terms, including $R^2$, are perturbatively
computable as:
\begin{equation}
\begin{array}{l}
\displaystyle H(Q) = 1+a_s(Q)H^{(1)}+a_s^2(Q)H^{(2)}+\mathcal{O}(a_s^3)\,,\\
\\
\displaystyle \mathcal{C}_{ik}(x,\mathcal{L}) = \delta_{ik}\delta(1-x)+a_s(Q)
  \mathcal{C}_{ik}^{(1)}(x,\mathcal{L}) +a_s^2(Q) \mathcal{C}_{ik}^{(2)}(x,\mathcal{L})
  +\mathcal{O}(a_s^3) \,,\\
\\
\displaystyle \mathbb{C}_{jl}(z,\mathcal{L}) = \delta_{jl}\delta(1-z)+a_s(Q) \mathbb{C}_{jl}^{(1)}(z,\mathcal{L}) +a_s^2(Q) \mathbb{C}_{jl}^{(2)}(z,\mathcal{L}) +\mathcal{O}(a_s^3) \,.
\end{array}
\end{equation}
I now need to put everything together and truncate to order $a_s^2$ in
such a way that:
\begin{equation}
\hat{B}_{ij,kl}(x,z;b) = \hat{B}_{ij,kl}^{(0)}(x,z;b)+a_s(Q) \hat{B}_{ij,kl}^{(1)}(x,z;b)+a_s^2(Q) \hat{B}_{ij,kl}^{(2)}(x,z;b)+\mathcal{O}(a_s^3) \,,
\end{equation}
with:
\begin{equation}\label{eq:pertcoefb}
\begin{array}{rcl}
  \displaystyle \hat{B}_{ij,kl}^{(0)} &=&\displaystyle
                                          \delta_{ik}\delta_{jl}\delta(1-x)\delta(1-z)\,,\\
  \\
  \displaystyle \hat{B}_{ij,kl}^{(1)} &=&\displaystyle
                                          (H^{(1)}+R^{(1)})\delta_{ik}\delta_{jl}\delta(1-x)\delta(1-z)\\
  \\
                                      &+& \mathcal{C}_{ik}^{(1)}(x,\mathcal{L})\delta_{jl}\delta(1-z) + \delta_{ik}\delta(1-x)\mathbb{C}_{jl}^{(1)}(z,\mathcal{L})\\
  \\
  \displaystyle \hat{B}_{ij,kl}^{(2)} &=& \displaystyle
                                          (H^{(2)}+H^{(1)}R^{(1)}+R^{(2)})
                                          \delta_{ik}\delta_{jl}\delta(1-x)\delta(1-z)\\
  \\
                                      &+&\displaystyle
                                          (H^{(1)}+R^{(1)})\left[\mathcal{C}_{ik}^{(1)}(x,\mathcal{L})\delta_{jl}\delta(1-z)
                                          +
                                          \delta_{ik}\delta(1-x)\mathbb{C}_{jl}^{(1)}(z,\mathcal{L})\right]\\
  \\
                                      &+& \mathcal{C}_{ik}^{(2)}(x,\mathcal{L})\delta_{jl}\delta(1-z) + 
                                          \mathcal{C}_{ik}^{(1)}(x,\mathcal{L})\mathbb{C}_{jl}^{(1)}(z,\mathcal{L})
                                          +\delta_{ik}\delta(1-x)\mathbb{C}_{jl}^{(2)}(z,\mathcal{L})
\end{array}
\end{equation}

Despite we derived the expansion of the resummed cross section up to
$\mathcal{O}(a_s^2)$, these are not yet the final formulas. The reason
is that, in order some of the terms of the expansion above do not
depend on the impact parameter $b$. This means that these terms, upon
Fourier transform needed to obtain the expression in the transverse
moment $q_T$, will give rise to terms proportional to
$\delta(q_T)$. In view of the matching procedure, since the
fixed-order expression we are going to match to does not include any
$\delta(q_T)$ terms, we need to make sure that they are not
subtracted.Therefore, we need to identify and remove any term that
does not depend on $b$. In fact, by construction, this is equivalent
to leave only terms proportional to a power of $\mathcal{L}$. In order
to do so, we start observing that the hard factor $H$ does not contain
any $\mathcal{L}$. In addition, we have to pay attention to the
coefficients $R^{(n)}$ because they are polynomials in $\mathcal{L}$
but also include a constant term ($\mathcal{L}^0$). Finally, the
$\zeta$-prescription, Eq.~(\ref{eq:zetapresc}), provides us with a
simple recipe to compute the logarithmic terms of the matching
functions $\mathcal{C}_{ik}^{(n)}$ and
$\mathbb{C}_{jl}^{(n)}$. Specifically, the condition of independence
of the TMD PDFs and FFs from the factorisation scale $\mu$ is such
that TMDs behave like physical observable (\textit{e.g.}
deep-inelastic-scattering or single-inclusive-annihilation structure
functions) and thus obey the standard scale variation rules derived,
for example, in Eq.~(2.17) of Ref.~\cite{vanNeerven:2000uj}. More in
particular, one finds that the matching function coefficients s have
the usual logarithmic structure:
\begin{equation}
\mathcal{C}_{ij}^{(n)}(x,\mathcal{L}) =
\sum_{k=0}^{n}\mathcal{C}_{ij}^{(n,k)}(x)\mathcal{L}^k\quad\mbox{and}\quad \mathbb{C}_{ij}^{(n)}(x,\mathcal{L}) = \sum_{k=0}^{n}\mathbb{C}_{ij}^{(n,k)}(x)\mathcal{L}^k \,,
\end{equation}
where the non-logarithmic terms $\mathcal{C}_{ij}^{(n,0)}$ and
$\mathbb{C}_{ij}^{(n,0)}$ have to be computed explicitly while the
other terms proportional to a positive power of $\mathcal{L}$ can be
expressed in terms of the non-logarithmic term of the previous orders
and of the coefficients of the DGLAP splitting functions and of the
QCD $\beta$ function:
\begin{equation}
\begin{array}{rcl}
  \mathcal{C}_{ij}^{(1,1)}(x) &=& -\mathcal{P}_{ij}^{(1)}(x) \,,\\
  \\
  \mathcal{C}_{ij}^{(2,1)}(x) &=& -\left(\mathcal{P}_{ij}^{(2)}(x)+\mathcal{C}_{ik}^{(1,0)}(x)
                                  \otimes
                                  \mathcal{P}_{kj}^{(1)}(x)-\beta_0
                                  \mathcal{C}_{ij}^{(1,0)}(x) \right)\,,\\
  \\
  \mathcal{C}_{ij}^{(2,2)}(x) &=&\displaystyle
                                  \frac12\left(\mathcal{P}_{ik}^{(1)}(x)\otimes \mathcal{P}_{kj}^{(1)}(x)-\beta_0 \mathcal{P}_{ij}^{(1)}(x)\right)\,,\\
\end{array}
\end{equation}
and:
\begin{equation}
\begin{array}{rcl}
  \mathbb{C}_{ij}^{(1,1)}(x) &=& -\mathbb{P}_{ij}^{(1)}(x) \,,\\
  \\
  \mathbb{C}_{ij}^{(2,1)}(x) &=& -\left(\mathbb{P}_{ij}^{(2)}(x)+\mathbb{C}_{ik}^{(1,0)}(x)
                                  \otimes
                                  \mathbb{P}_{kj}^{(1)}(x)-\beta_0
                                  \mathbb{C}_{ij}^{(1,0)}(x) \right)\,,\\
  \\
  \mathbb{C}_{ij}^{(2,2)}(x) &=&\displaystyle
                                  \frac12\left(\mathbb{P}_{ik}^{(1)}(x)\otimes \mathbb{P}_{kj}^{(1)}(x)-\beta_0 \mathbb{P}_{ij}^{(1)}(x)\right)\,,\\
\end{array}
\end{equation}
where $\mathcal{P}_{ij}^{(n)}$ and $\mathbb{P}_{ij}^{(n)}$ are the
coefficients of the $a_s^n$ terms of the space- and time-like
splitting functions, respectively. With this information at hand,
knowing the logarithmic expansion of the coefficients $R^{(n)}$, and
keeping in mind that the hard coefficients $H^{(n)}$ do not contain
any logarithms, we can organize the coefficients
inEq.~(\ref{eq:pertcoefb}) in terms of powers of
$\mathcal{L}$. Specifically, we find that:
\begin{equation}
\hat{B}_{ij,kl}^{(n)} = \sum_{p=0}^{2n}\hat{B}_{ij,kl}^{(n,p)}\mathcal{L}^p\,,
\end{equation}
with the $\mathcal{O}(1)$ coefficient being:
\begin{equation}
\begin{array}{rcl}
  \hat{B}_{ij,kl}^{(0,0)} &=& \delta_{ik}\delta_{kl}\delta(1-x)
                              \delta(1-z)\,,
\end{array}
\end{equation}
the $\mathcal{O}(a_s)$ coefficients being:
\begin{equation}
\begin{array}{rcl}
  \hat{B}_{ij,kl}^{(1,0)} &=&\displaystyle
                              (H^{(1)}+S^{(1,0)})\delta_{ik}\delta_{jl}\delta(1-x)\delta(1-z)+ \mathcal{C}_{ik}^{(1,0)}(x)\delta_{jl}\delta(1-z) + \delta_{ik}\delta(1-x)\mathbb{C}_{jl}^{(1,0)}(z)\,,\\
  \\
  \hat{B}_{ij,kl}^{(1,1)} &=& S^{(1,1)}\delta_{ik}\delta_{jl}\delta(1-x)\delta(1-z)+ \mathcal{C}_{ik}^{(1,1)}(x)\delta_{jl}\delta(1-z) + \delta_{ik}\delta(1-x)\mathbb{C}_{jl}^{(1,1)}(z) \,,\\
  \\
  \hat{B}_{ij,kl}^{(1,2)} &=&
                              S^{(1,2)}\delta_{ik}\delta_{jl}\delta(1-x)\delta(1-z)
\end{array}
\end{equation}
and the $\mathcal{O}(a_s^2)$ coefficients being:
\begin{equation}
\begin{array}{rcl}
  \hat{B}_{ij,kl}^{(2,0)} &=&\displaystyle (H^{(2)}+H^{(1)}S^{(1,0)}+\widetilde{S}^{(2,0)})
                              \delta_{ik}\delta_{jl}\delta(1-x)\delta(1-z) \,,\\
  \\
                          &+&\displaystyle
                              (H^{(1)}+S^{(1,0)})\left[\mathcal{C}_{ik}^{(1,0)}(x)\delta_{jl}\delta(1-z)
                              +
                              \delta_{ik}\delta(1-x)\mathbb{C}_{jl}^{(1,0)}(z)\right]\\
  \\
                          &+&\displaystyle \mathcal{C}_{ik}^{(2,0)}(x)\delta_{jl}\delta(1-z) + 
                              \mathcal{C}_{ik}^{(1,0)}(x)\mathbb{C}_{jl}^{(1,0)}(z)
                              +\delta_{ik}\delta(1-x)\mathbb{C}_{jl}^{(2,0)}(z)\\

  \\
  \hat{B}_{ij,kl}^{(2,1)} &=&\displaystyle (H^{(1)}S^{(1,1)}+\widetilde{S}^{(2,1)})
                              \delta_{ik}\delta_{jl}\delta(1-x)\delta(1-z)\\
  \\
                          &+& \displaystyle
                              (H^{(1)}+S^{(1,0)})\left[\mathcal{C}_{ik}^{(1,1)}(x)\delta_{jl}\delta(1-z)
                              +
                              \delta_{ik}\delta(1-x)\mathbb{C}_{jl}^{(1,1)}(z)\right]\\
  \\
                          &+&
                              S^{(1,1)}\left[\mathcal{C}_{ik}^{(1,0)}(x)\delta_{jl}\delta(1-z)
                              +
                              \delta_{ik}\delta(1-x)\mathbb{C}_{jl}^{(1,0)}(z)\right]\\
  \\
                          &+& \mathcal{C}_{ik}^{(2,1)}(x)\delta_{jl}\delta(1-z) + 
                              \mathcal{C}_{ik}^{(1,1)}(x)\mathbb{C}_{jl}^{(1,0)}(z) +\mathcal{C}_{ik}^{(1,0)}(x)\mathbb{C}_{jl}^{(1,1)}(z)
                              +\delta_{ik}\delta(1-x)\mathbb{C}_{jl}^{(2,1)}(z)\\
  \\
  \hat{B}_{ij,kl}^{(2,2)} &=&\displaystyle (H^{(1)}S^{(1,2)}+\widetilde{S}^{(2,2)})
                              \delta_{ik}\delta_{jl}\delta(1-x)\delta(1-z)\\
  \\
                          &+&\displaystyle
                              \widetilde{S}^{(1,2)}\left[\mathcal{C}_{ik}^{(1,0)}(x)\delta_{jl}\delta(1-z)+\delta_{ik}\delta(1-x)\mathbb{C}_{jl}^{(1,0)}(z)\right]\\
  \\
                          &+&\displaystyle
                              S^{(1,1)}\left[\mathcal{C}_{ik}^{(1,1)}(x)\delta_{jl}\delta(1-z)+\delta_{ik}\delta(1-x)\mathbb{C}_{jl}^{(1,1)}(z)\right]\\
  \\
                          &+& \mathcal{C}_{ik}^{(2,2)}(x)\delta_{jl}\delta(1-z) + 
                              \mathcal{C}_{ik}^{(1,1)}(x)\mathbb{C}_{jl}^{(1,1)}(z)
                              +\delta_{ik}\delta(1-x)\mathbb{C}_{jl}^{(2,2)}(z)\\
  \\
  \hat{B}_{ij,kl}^{(2,3)} &=&\displaystyle
                              \widetilde{S}^{(2,3)}\delta_{ik}\delta_{jl}\delta(1-x)\delta(1-z)
                              + S^{(1,2)}\left[\mathcal{C}_{ik}^{(1,1)}(x)\delta_{jl}\delta(1-z)
                              +
                              \delta_{ik}\delta(1-x)\mathbb{C}_{jl}^{(1,1)}(z)\right]
  \\
  \\
  \hat{B}_{ij,kl}^{(2,4)} &=&\displaystyle \widetilde{S}^{(2,4)}\delta_{ik}\delta_{jl}\delta(1-x)\delta(1-z)
\end{array}
\end{equation}

In order to obtain a differential cross section in $q_T$, we need to
take the Fourier transform of $\hat{B}_{ij,kl}$, that is:
\begin{equation}
 \hat{B}_{ij,kl}(x,z;q_T) \equiv \int\frac{d^2\mathbf{b}}{4\pi} e^{i \mathbf{b}\cdot
  \mathbf{q}_T} \hat{B}_{ij,kl}(x,z;b) = \sum_{n=0}^2a_s^n(Q)\sum_{p=0}^{2n}\hat{B}_{ij,kl}^{(n,p)}(x,z) I_p(q_T)\,,
\end{equation}
where I have defined:
\begin{equation}
I_p(q_T) = \int\frac{d^2\mathbf{b}}{4\pi} e^{i \mathbf{b}\cdot
  \mathbf{q}_T}\mathcal{L}^p=\int\frac{d^2\mathbf{b}}{4\pi} e^{i \mathbf{b}\cdot
  \mathbf{q}_T}\ln^p\left(\frac{b^2Q^2}{C_0^2}\right) = \frac12\int_0^\infty db\,b J_0(bq_T) \ln^p\left(\frac{b^2Q^2}{C_0^2}\right)\,.
\end{equation}
Results for $I_p$ have been computed up to $p=4$ in Eq.~(136) of
Appendix B of Ref.~\cite{Bozzi:2005wk}. Specifically, and including
the trivial tranform with $p=0$, they read:
\begin{equation}
\begin{array}{l}
\displaystyle I_0(q_T) = \delta(q_T)\,,\\
\\
\displaystyle I_1(q_T) = - \frac{1}{q_T^2}\,,\\
\\
\displaystyle I_2(q_T) = - \frac{2}{q_T^2}\ln\left(\frac{Q^2}{q_T^2}\right) \,,\\
\\
\displaystyle I_3(q_T) = - \frac{3}{q_T^2}\ln^2\left(\frac{Q^2}{q_T^2}\right) \,,\\
\\
\displaystyle I_4(q_T) = - \frac{4}{q_T^2}\left[\ln^3\left(\frac{Q^2}{q_T^2}\right)-4\zeta_3\right]\,.
\end{array}
\end{equation}
As clear from the transforms above, all terms with $p=0$ will be
proportional to $\delta(q_T)$. We do not need to consider these terms
because analogous terms are not included in the fixed-order
calculation and thus does not need to be subtracted. Therefore, we write:
\begin{equation}\label{eq:finalformula}
  \hat{B}_{ij,kl}(x,z;q_T) =
  \sum_{n=1}^2a_s^n(Q)\sum_{p=1}^{2n}\hat{B}_{ij,kl}^{(n,p)}(x,z)
  I_p(q_T)+ \left(\sum_{n=0}^2a_s^n(Q)\hat{B}_{ij,kl}^{(n,0)}(x,z)\right)\delta(q_T)+\mathcal{O}(a_s^3)\,,
\end{equation}
and we are not going to consider the term proportional to
$\delta(q_T)$, even though all terms have been derived above.

As clear from Eq.~(\ref{eq:finalformula}), removing all terms
proportional to $\delta(q_T)$ also means removing the full
$\mathcal{O}(1)$ terms such that leading-order term is now
$\mathcal{O}(a_s)$.

In order to validate the results above, it is opportune to compare the
$\mathcal{O}(a_s)$ expressions to those present in the literature. To
this end, we write explicitly the expression for
$\hat{B}_{ij,kl}^{(1)}$ in $q_T$ space without the $\delta(q_T)$ term:
\begin{equation}
\begin{array}{rcl}
\hat{B}_{ij,kl}^{(1)}(x,z;q_T) &=&\displaystyle 
-\hat{B}_{ij,kl}^{(1,1)}(x,z)\mathcal{L}\frac{1}{q_T^2}
  -\hat{B}_{ij,kl}^{(1,2)}(x,z)\frac{2}{q_T^2}\ln\left(\frac{Q^2}{q_T^2}\right)\\
\\
 &=&\displaystyle
     \frac{1}{q_T^2}\bigg[4C_F\left(\ln\left(\frac{Q^2}{q_T^2}\right)-\frac{3}{2}\right)\delta_{ik}\delta_{jl}\delta(1-x)\delta(1-z)\\
\\
 &+&\displaystyle \mathcal{P}_{ik}^{(1)}(x)\delta_{jl}\delta(1-z) + \delta_{ik}\delta(1-x)\mathbb{P}_{jl}^{(1)}(z)\bigg]\,,
\end{array}
\end{equation}
so that:
\begin{equation}
\begin{array}{rcl}
B_{ij}(x,z;q_T) &=&\displaystyle a_s(Q) \sum_{kl}\hat{B}_{ij,kl}^{(1)}\mathop{\otimes}_x
  f_k(x,Q)\mathop{\otimes}_zd_l(z,Q)+\mathcal{O}(a_s^2)\\
\\
&=&\displaystyle a_s(Q) \frac{1}{q_T^2}\bigg[4C_F\left(\ln\left(\frac{Q^2}{q_T^2}\right)-\frac{3}{2}\right) f_i(x,Q)d_j(z,Q)\\
\\
 &+&\displaystyle \left(\sum_{k}\mathcal{P}_{ik}^{(1)}(x) \mathop{\otimes}_x
  f_k(x,Q)\right)d_j(z,Q) + f_i(x,Q)\left(\sum_l\mathbb{P}_{jl}^{(1)}(z) \mathop{\otimes}_zd_l(z,Q)\right) +\mathcal{O}(a_s^2)\,.
\end{array}
\end{equation}
This result nicely agrees with that of, \textit{e.g.},
Refs.~\cite{Meng:1995yn,Collins:2016hqq}.





\newpage










Let us start from Eq.~(2.6) of Ref.~\cite{Scimemi:2017etj}, that is
the fully differential cross section for lepton-pair production in the
region in which the TMD factorisation applies, $i.e.$ $q_T \ll
Q$. After some minor manipulations, it reads:
\begin{equation}\label{eq:crosssection}
\frac{d\sigma}{dQ dy dq_T} =
\frac{16\pi\alpha^2(Q)q_T\mathcal{P}(q_T,Q)}{3N_c Q^3} H(Q,\mu) \sum_q C_q(Q)
\int\frac{d^2\mathbf{b}}{4\pi} e^{i \mathbf{b}\cdot \mathbf{q}_T} x_1F_q(x_1,\mathbf{b};\mu,\zeta) x_2F_{\bar{q}}(x_2,\mathbf{b};\mu,\zeta)\,,
\end{equation}
where $Q$, $y$, and $q_T$ are the invariant mass, the rapidity, and
the transverse momentum of the lepton pair, respectively, while
$N_c=3$ is the number of colours, $\alpha$ is the electromagnetic
coupling, $H$ is the appropriate QCD form factor that can be
perturbatively computed, and $C_q$ are the effective electroweak
charges. In addition, the variables $x_1$ and $x_2$ are functions of
$Q$ and $y$ and are given by:
\begin{equation}\label{eq:Bjorkenx12}
  x_{1,2} = \frac{Q}{\sqrt{s}}e^{\pm y}\,,
\end{equation}
being $\sqrt{s}$ the centre-of-mass energy of the collision. The
kinematic factor $\mathcal{P}$ takes into account the reduction of the
integration leptonic phase space due to possible cuts on the leptons
and thus it depends on $q_T$, $y$, and $Q$ as well as on the numerical
values of the cut parameters. Finally, the scales $\mu$ and $\zeta$
are introduced through TMD factorisation to factorise collinear and
rapidity divergences. As usual, despite they are arbitrary scales,
they are typically chosen $\mu=\sqrt{\zeta}=Q$. Therefore, for all
practical purposes their presence is fictitious.

The computation-intensive part of eq.(\ref{eq:crosssection}) has the
form of the integral:
\begin{equation}\label{eq:integral}
I_{ij}(x_1,x_2,q_T;\mu,\zeta)=\int\frac{d^2\mathbf{b}}{4\pi} e^{i \mathbf{b}\cdot \mathbf{q}_T} x_1F_i(x_1,\mathbf{b};\mu,\zeta) x_2F_{j}(x_2,\mathbf{b};\mu,\zeta)\,.
\end{equation}
where $F_{i(j)}$ are combinations of evolved TMD PDFs. At this stage,
for convenience, $i$ and $j$ do not coincide with $q$ and $\bar{q}$
but they are linked through a simple linear transformation. The
integral over the bidimensional impact parameter \textbf{b} has to be
taken. However, $F_{i(j)}$ only depend on the absolute value of
\textbf{b}, therefore eq.~(\ref{eq:integral}) can be written as:
\begin{equation}\label{eq:integral2}
I_{ij}(x_1,x_2,q_T;\mu,\zeta)=\frac12\int_0^\infty db\,b J_0(bq_T)  x_1
F_i(x_1,b;\mu,\zeta) x_2 F_{j}(x_2,b;\mu,\zeta)\,.
\end{equation}
where $J_0$ is the zero-th order Bessel function of the first kind
whose integral representation is:
\begin{equation}
J_0(x) = \frac1{2\pi}\int_0^{2\pi} d\theta e^{ix\cos(\theta)}\,.
\end{equation}
The single evolved TMD PDF $F_i$ at the final scales $\mu$ and $\zeta$
is obtained by multiplying the same TMD PDF at the initial scales
$\mu_0$ and $\zeta_0$ by a single evolution factor
$R_q$(\footnote{Note that in eq.~(\ref{eq:crosssection}) the gluon TMD
  PDF $F_g$ is not involved. If also the gluon TMD PDF was involved,
  it would evolve by means of a different evolution factor $R_g$.}),
that is:
\begin{equation}
  xF_i(x,b;\mu,\zeta) = R_q(\mu_0,\zeta_0\rightarrow \mu,\zeta;b)
  xF_i(x,b;\mu_0,\zeta_0)\,.
\end{equation}
so that eq.~(\ref{eq:integral2}) becomes:
\begin{equation}\label{eq:integral3}
I_{ij}(x_1,x_2,q_T;\mu,\zeta)=\frac12\int_0^\infty db\,b J_0(bq_T)
\left[R_q(\mu_0,\zeta_0\rightarrow \mu,\zeta;b)\right]^2 x_1 F_i(x_1,b;\mu_0,\zeta_0) x_2F_{j}(x_2,b;\mu_0,\zeta_0)\,.
\end{equation}

The initial scale TMD PDFs at LO in the OPE region, that is for
$b\ll B$ where $B$ is an unknown non-perturbative parameter that
represents the intrinsic hadron scale (see eq.~(2.27) of
Ref.~\cite{Scimemi:2017etj}), can be written as:
\begin{equation}\label{eq:LOconv}
xF_i(x,b;\mu_0,\zeta_0) = \sum_{j=g,q(\bar{q})}x\int_x^1\frac{dy}{y}C_{ij}(y;\mu_0,\zeta_0)f_j\left(\frac{x}{y},\mu_0\right)\,,
\end{equation}
where $f_j$ are the collinear PDFs (including the gluon) and $C_{ij}$
are the so-called matching functions that are perturbatively
computable and are currently known to NNLO, $i.e.$
$\mathcal{O}(\alpha_s^2)$. If we define:
\begin{equation}
\widetilde{f}_i\left(x,\mu_0\right) = xf_i\left(x,\mu_0\right)\,,
\end{equation}
eq.~(\ref{eq:LOconv}) can be written as:
\begin{equation}\label{eq:LOconvNPbl}
x F_i(x,b;\mu_0,\zeta_0) =
\sum_{j=g,q(\bar{q})}\int_x^1dy\,C_{ij}(y;\mu_0,\zeta_0)  \widetilde{f}_i\left(\frac{x}{y},\mu_0\right)\,.
\end{equation}
At this point, it is opportune to mention that the variables $\mu_0$
and $\zeta_0$ are usually taken to be functions of the impact
parameter $b$. Therefore, eq.~(\ref{eq:LOconvNPbl}) is a function of
two variables only that we rewrite way as:
\begin{equation}\label{eq:LOconvNPsimp}
  x F_i(x,b) =
  \sum_{j=g,q(\bar{q})}\int_x^1dy\,C_{ij}(y,b)  \widetilde{f}_i\left(\frac{x}{y},b\right)\,.
\end{equation}

This kind of convolutions can be computed using standard interpolation
techniques by which one approximates the function $\widetilde{f}_i$ as:
\begin{equation}
  \widetilde{f}_i(x,b) = \sum_\alpha w_\alpha(x) \widetilde{f}_i(x_\alpha,b)
\end{equation}
where $x_\alpha$ is the $\alpha$-th node of an interpolation grid and
$w_\alpha$ is the interpolating function associated to that
node. Assuming for now that $x$ coincides with the $\beta$-th node of
the grid and introducing another grid in the $b$ dimension whose nodes
are indexed by $\tau$, eq.~(\ref{eq:LOconvNPsimp}) can be written as:
\begin{equation}\label{eq:LOconvNPsimp1}
  \hat{F}_{i,\beta}^\tau = \sum_j\sum_\alpha\hat{C}_{ij,\beta\alpha}^\tau \hat{f}_{j,\alpha}^\tau\,.
\end{equation}
where we have used the following definitions:
\begin{equation}
\hat{F}_{i,\beta}^\tau\equiv x_\beta F_i(x_\beta,b_\tau)\,,\quad
\hat{C}_{ij,\beta\alpha}^\tau \equiv
\int_{x_\beta}^1dy\,C_{ij}(y,b_\tau)w_\alpha\left(\frac{x_\beta}{y}\right)\,,\quad
  \hat{f}_{j,\alpha}^\tau \equiv \widetilde{f}_i(x_\alpha,b_\tau)\,.
\end{equation}

Since we have to integrate over the impact parameter $b$ (see
eq.~(\ref{eq:integral2})), we need to be able to reconstruct the \
dependence of the function $F_i$ on $b$. This can be done using the
same interpolation technique. In particular, we write:
\begin{equation}
  x_\alpha F_i(x_{\alpha},b)
  x_\beta F_j(x_\beta,b) = 
  \sum_\tau \widetilde{w}_{\tau}(b) \hat{F}_{i,\alpha}^\tau \hat{F}_{j,\beta}^\tau
  = \sum_\tau \widetilde{w}_{\tau}(b) \sum_{kl}\sum_{\gamma\delta} \hat{C}_{ik,\alpha\gamma}^\tau
  \hat{C}_{jl,\beta\delta}^\tau \hat{f}_{k,\gamma}^\tau \hat{f}_{l,\delta}^\tau\,.
\end{equation}
Keeping in mind that $\mu_0$ and $\zeta_0$ are functions of the impact
parameter $b$ and that $\mu=\sqrt{\zeta} =Q$, eq.~(\ref{eq:integral3})
takes the form:
\begin{equation}\label{eq:lumiInter}
  I_{ij}(x_\alpha,x_\beta,q_T;Q)= \sum_\tau K_\tau(Q;q_T)
  \sum_{kl}\sum_{\gamma\delta} \hat{C}_{ik,\alpha\gamma}^\tau
  \hat{C}_{jl,\beta\delta}^\tau \hat{f}_{k,\gamma}^\tau \hat{f}_{l,\delta}^\tau\,,
\end{equation}
where we have defined:
\begin{equation}\label{eq:Kcoeff}
  K_\tau(Q;q_T)\equiv\frac12\int_0^\infty db\,b J_0(bq_T)
  \left[R_q(Q;b)\right]^2 \widetilde{w}_{\tau}(b)\,,
\end{equation}
being $R_q(Q;b)\equiv R_q(\mu_0,\zeta_0\rightarrow \mu,\zeta;b)$. It
should be noticed that $\widetilde{w}_{\tau}$ is a piecewise function
different from zero only over a finite interval in $b$, say
$[c_\tau,d_\tau]$. In practice, $\widetilde{w}_{\tau}$ extends over
$k+1$ intervals on the grid in $b$, being $k$ the interpolation
degree, around the node $b_\tau$ so that, typically
$c_\tau=b_{\tau-k}$ and $d_\tau=b_{\tau+1}$. Therefore the integral in
eq.~(\ref{eq:Kcoeff}) reduces to:
\begin{equation}\label{eq:Kcoeff1}
  K_\tau(Q;q_T)\equiv\frac12\int_{c_\tau}^{d_\tau} db\,b J_0(bq_T)
  \left[R_q(Q;b)\right]^2 \widetilde{w}_{\tau}(b)\,,
\end{equation}
and the sum over $\tau$ in eq.~(\ref{eq:lumiInter}), that is supposed
to run over an infinite number of nodes, has to be truncated.

As customary in QCD, the most convenient flavour basis, that is the
one that minimises the mixing between operators, is the so-called
``evolution'' basis (\textit{i.e.} $\Sigma$, $V$, $T_3$, $V_3$,
etc.). In fact, in this basis the operators matrix $C_{ij}$ is almost
diagonal with the only exception of crossing terms that couple the
gluon and the singlet $\Sigma$ distributions. This greatly simplifies
the sums over $k$ and $l$ in eq.~(\ref{eq:lumiInter}). On the other
hand, given that the TMDs that appear in eq.~(\ref{eq:crosssection})
are in the so-called ``physical'' basis (\textit{i.e.} $d$, $\bar{d}$,
$u$, $\bar{u}$, etc.), we need to rotate the quantity in
eq.~(\ref{eq:lumiInter}) from the evolution basis, over which the
indices $i$ and $j$ run, to the physical basis. This is done by means
of an appropriate constant matrix $T$, so that:
\begin{equation}\label{eq:lumiInterRot}
I_{q\bar{q}}(x_\alpha,x_\beta,q_T;Q)= \sum_\tau \sum_{kl} \sum_{\gamma\delta}\sum_{ij}K_\tau(Q;q_T) T_{qi}T_{\bar{q}j}\hat{C}_{ik,\alpha\gamma}^\tau
  \hat{C}_{jl,\beta\delta}^\tau \hat{f}_{k,\gamma}^\tau \hat{f}_{l,\delta}^\tau\,.
\end{equation}

In order account for higher orders in the OPE and non-perturbative
effects where the OPE is not valid, one usually introduces a
phenomenological non-perturbative function $f_{\rm NP}$ that modifies
the convolution in eq.~(\ref{eq:LOconv}) The way how $f_{\rm NP}(x,b)$
is introduced in not unique. Here we choose to follow the most
traditional approach in which TMDs get corrected by a multiplicative
function, that is to say:
\begin{equation}\label{eq:LOconvNP1}
  xF_i(x,b) \rightarrow f_{\rm NP}(x,b) xF_i(x,b)\,.
\end{equation}
This can be easily introduced in eq.~(\ref{eq:x1x2inter}) by defining:
\begin{equation}
f_{{\rm NP},\alpha}^\tau \equiv f_{\rm NP}(x_\alpha,b_\tau)\,,
\end{equation}
so that:
\begin{equation}\label{eq:lumiInterRotNP}
I_{q\bar{q}}(x_\alpha,x_\beta,q_T;Q)= \sum_\tau \sum_{kl} \sum_{\gamma\delta}\sum_{ij}K_\tau(Q;q_T) T_{qi}T_{\bar{q}j}\hat{C}_{ik,\alpha\gamma}^\tau
  \hat{C}_{jl,\beta\delta}^\tau \hat{f}_{k,\gamma}^\tau \hat{f}_{l,\delta}^\tau f_{{\rm NP},\alpha}^\tau f_{{\rm NP},\beta}^\tau\,.
\end{equation}
The computation of $I_{q\bar{q}}$ for a generic $x_1$ and $x_2$ is
achieved by interpolation as:
\begin{equation}\label{eq:x1x2inter}
\begin{array}{c}
  \displaystyle I_{q\bar{q}}(x_1,x_2,q_T;Q) =
  \sum_{\alpha\beta}w_\alpha(x_1)w_\beta(x_2)I_{q\bar{q}}(x_\alpha,x_\beta,q_T;Q)
  =\\
\\
\displaystyle \sum_\tau\sum_{\alpha\beta} \sum_{kl} \sum_{\gamma\delta}\sum_{ij}K_\tau(Q;q_T)w_\alpha(x_1)w_\beta(x_2) T_{qi}T_{\bar{q}j}\hat{C}_{ik,\alpha\gamma}^\tau
  \hat{C}_{jl,\beta\delta}^\tau \hat{f}_{k,\gamma}^\tau \hat{f}_{l,\delta}^\tau f_{{\rm NP},\alpha}^\tau f_{{\rm NP},\beta}^\tau\,.
\end{array}
\end{equation}

Keeping in mind eq.~(\ref{eq:Bjorkenx12}), one realises that the
variables $x_1$ and $x_2$ are functions of $Q$ and $y$ and thus one
can simply write:
\begin{equation}\label{eq:finalres}
  I_{q\bar{q}}(Q, y, q_T) =
  \sum_\tau \sum_{\alpha\beta} W_{q\bar{q},\alpha\beta}^{\tau}(Q,y,q_T) f_{{\rm NP},\alpha}^\tau f_{{\rm NP},\beta}^\tau\,,
\end{equation}
where we have defined:
\begin{equation}
W_{q\bar{q},\alpha\beta}^{\tau}(Q,y,q_T)\equiv \sum_{kl} \sum_{\gamma\delta}\sum_{ij}K_\tau(Q;q_T) w_\alpha\left(\frac{Q}{\sqrt{s}}e^y\right)w_\beta\left(\frac{Q}{\sqrt{s}}e^{-y}\right) T_{qi}T_{\bar{q}j}\hat{C}_{ik,\alpha\gamma}^\tau
  \hat{C}_{jl,\beta\delta}^\tau \hat{f}_{k,\gamma}^\tau \hat{f}_{l,\delta}^\tau\,.
\end{equation}
Unsurprisingly, the $W$ factors can be factorised as:
\begin{equation}
  W_{q\bar{q},\alpha\beta}^{\tau}(Q,y,q_T)\equiv
  K_\tau(Q;q_T) w_\alpha\left(\frac{Q}{\sqrt{s}}e^y\right) \left(\sum_{i} T_{qi}\sum_{k} \sum_{\gamma}\hat{C}_{ik,\alpha\gamma}^\tau
    \hat{f}_{k,\gamma}^\tau\right)
w_\beta\left(\frac{Q}{\sqrt{s}}e^{-y}\right)  \left(\sum_{j} T_{\bar{q}j}\sum_{l} \sum_{\delta}
    \hat{C}_{jl,\beta\delta}^\tau \hat{f}_{l,\delta}^\tau\right)\,.
\end{equation}

This equation emphasises that $I_{q\bar{q}}$ is a function of three
independent kinematics variables $Q$, $y$, and $q_T$. This is relevant
when integrating the cross section over the experimental bins as we
will discuss in the next section. With eq.~(\ref{eq:finalres}) at
hand, eq.~(\ref{eq:crosssection}) can be written as:
\begin{equation}
\frac{d\sigma}{dQ dy dq_T} =\sum_\tau \sum_{\alpha\beta}\left[
  \frac{16\pi\alpha^2(Q)q_T\mathcal{P}(q_T,Q)}{3N_c Q^3} H(Q) 
  \sum_q C_q(Q) W_{q\bar{q},\alpha\beta}^{\tau}(Q,y,q_T) \right]f_{{\rm NP},\alpha}^\tau f_{{\rm NP},\beta}^\tau\,.
\end{equation}
Crucially, the quantity inside the squared brackets is fully
determined by the kinematics and the leading-twist component of the
process, while the non-perturbative part is fully factorised. Clearly,
this is extremely useful if one wants to fit the non-perturbative
component to data.

\section{Integrating over the final-state kinematic variables}

Despite eq.~(\ref{eq:lumiInterRot}) provides a powerful tool for a
fast computation of cross sections, it is often not sufficient to
allow for a direct comparison to experimental data. The reason is that
experimental measurements of differential distributions are usually
delivered integrated over finite regions of the final-state kinematic
phase space. In other words, experiments measure quantities like:
\begin{equation}\label{eq:Intcrosssection}
\widetilde{\sigma}=\int_{Q_{\rm min}}^{Q_{\rm max}}dQ \int_{y_{\rm min}}^{y_{\rm max}}dy \int_{q_{T,\rm min}}^{q_{T,\rm max}}dq_T\left[\frac{d\sigma}{dQ dy dq_T} \right]\,.
\end{equation}
As a consequence, in order to guarantee performance, we need to
include the integrations above in the precomputed factors.

\subsection{Integrating over $q_T$}

The integration over $q_T$ is relatively simple to implement because
the full dependence on $q_T$ in eq.~(\ref{eq:Kcoeff1}) is given by the
factors $q_T$, $\mathcal{P}$, and $K_\tau$. Therefore, integrating
over $q_T$ simply amounts of computing the integrals:
\begin{equation}\label{eq:KcoeffInt}
  \widetilde{K}_\tau(Q)\equiv \int_{q_{T,\rm min}}^{q_{T,\rm max}}dq_T\,q_T\mathcal{P}(q_T,Q) K_\tau(Q;q_T)
\end{equation}

\subsection{Integrating over $y$}

The dependence on $y$ of the cross section in
eq.~(\ref{eq:crosssection}) exclusively happens through the variables
$x_1$ and $x_2$ defined in eq.~(\ref{eq:Bjorkenx12}). Since this
dependence is reconstructed through interpolation in
eq.~(\ref{eq:x1x2inter}), what we need to do is computing the
following integrals:
\begin{equation}
u_{\alpha\beta}(Q)\equiv \int_{y_{\rm min}}^{y_{\rm max}}dy\,w_\alpha\left(\frac{Q}{\sqrt{s}}e^y\right)w_\beta\left(\frac{Q}{\sqrt{s}}e^{-y}\right)
\end{equation}
and replace $w_\alpha(x_1)w_\beta(x_2)$ in eq.~(\ref{eq:x1x2inter})
with $u_{\alpha\beta}(Q)$.

\subsection{Integrating over $Q$}

The integration over $Q$ has finally to be done by brute force due to
the fact that the dependence on $Q$ of the expression we are
considering is not localised and involves essentially all ingredients
(we remind that we are assuming $\mu=\sqrt{\zeta}=Q$). One alternative
solution is to use the so-called narrow-width approximation (NWA) in
which one assumes that the width of the $Z$ boson $\Gamma_Z$ is much
smaller that its mass $M_Z$. This way one can approximate the peaked
behaviour of the couplings $C_q(Q)$ around $Q=M_Z$ with a
$\delta$-function, \textit{i.e.} $C_q(Q)\sim \delta(Q^2-M_Z^2)$, so
that the integration over $Q$ can be done analytically essentially
setting $Q=M_Z$ everywhere in the expression. This approximation,
though, is usable only for data around the $Z$ peak and, of course, it
is only an approximation and thus might lead to substantial
inaccuracies. Therefore, it is useful to be able to carry out the
integration over $Q$ explicitly.

To this end, we start by writing explicitly the cross section
integrated over $q_T$ and $y$ making use of the definitions given in
the previous subsections:
\begin{equation}
  \frac{d\sigma}{dQ} =\sum_\tau \sum_{\alpha\beta}\left[\frac{16\pi}{3N_c}\sum_{kl} \sum_{\gamma\delta}\sum_{ij}\sum_q
    \left(\frac{\alpha^2(Q)}{Q^3} H(Q) u_{\alpha\beta}(Q)
    \widetilde{K}_\tau(Q) C_q(Q)\right)
    T_{qi}T_{\bar{q}j}
    \hat{C}_{ik,\alpha\gamma}^\tau \hat{C}_{jl,\beta\delta}^\tau 
  \hat{f}_{k,\gamma}^{\tau}
  \hat{f}_{l,\delta}^{\tau}\right]f_{{\rm NP},\alpha}^\tau f_{{\rm NP},\beta}^\tau\,,
\end{equation}
where we have purposely enclosed between round brakets the
$Q$-dependant factors. In fact, if we define:
\begin{equation}
S_{q,\alpha\beta}^{\tau}\equiv \frac{16\pi}{3N_c}\int_{Q_{\rm min}}^{Q_{\rm max}}dQ\,
    \frac{\alpha^2(Q)}{Q^3} H(Q) \widetilde{K}_\tau(Q) u_{\alpha\beta}(Q)
    C_q(Q)\,,
\end{equation}
we have that:
\begin{equation}\label{eq:FullInt}
  \widetilde{\sigma} =\sum_\tau \sum_{\alpha\beta}\left[ \sum_q
    S_{q,\alpha\beta}^{\tau} \left(\sum_{i}
    T_{qi}\sum_{k} \sum_{\gamma}
    \hat{C}_{ik,\alpha\gamma}^\tau 
  \hat{f}_{k,\gamma}^{\tau}\right)
\left(\sum_{j}
    T_{\bar{q}j}\sum_{l} \sum_{\delta}
    \hat{C}_{jl,\beta\delta}^\tau 
  \hat{f}_{l,\delta}^{\tau}\right)
\right]f_{{\rm NP},\alpha}^\tau f_{{\rm NP},\beta}^\tau\,.
\end{equation}
so that, defining:
\begin{equation}
  \overline{F}_{q(\bar{q}),\alpha}^\tau \equiv \sum_{i}
  T_{q(\bar{q})i}\sum_{k} \sum_{\gamma}
  \hat{C}_{ik,\alpha\gamma}^\tau 
  \hat{f}_{k,\gamma}^{\tau}\,,
\end{equation}
eq.~(\ref{eq:FullInt}) can be readily written as:
\begin{equation}
  \widetilde{\sigma} =\sum_\tau \sum_{\alpha\beta}\left[ \sum_q
    S_{q,\alpha\beta}^{\tau}\overline{F}_{q,\alpha}^\tau
    \overline{F}_{\bar{q},\beta}^\tau \right]f_{{\rm NP},\alpha}^\tau
  f_{{\rm NP},\beta}^\tau=\sum_\tau
  \sum_{\alpha\beta}M_{\alpha\beta}^\tau f_{{\rm NP},\alpha}^\tau f_{{\rm NP},\beta}^\tau\,,
\end{equation}
with:
\begin{equation}
M_{\alpha\beta}^\tau\equiv \sum_q S_{q,\alpha\beta}^{\tau}\overline{F}_{q,\alpha}^\tau
\overline{F}_{\bar{q},\beta}^\tau\,.
\end{equation}



\begin{thebibliography}{alp}

%\cite{Scimemi:2017etj}
\bibitem{Scimemi:2017etj}
  I.~Scimemi and A.~Vladimirov,
  %``Analysis of vector boson production within TMD factorization,''
  arXiv:1706.01473 [hep-ph].
  %%CITATION = ARXIV:1706.01473;%%
  %2 citations counted in INSPIRE as of 24 Oct 2017

%\cite{Echevarria:2016scs}
\bibitem{Echevarria:2016scs}
  M.~G.~Echevarria, I.~Scimemi and A.~Vladimirov,
  %``Unpolarized Transverse Momentum Dependent Parton Distribution and Fragmentation Functions at next-to-next-to-leading order,''
  JHEP {\bf 1609} (2016) 004
  doi:10.1007/JHEP09(2016)004
  [arXiv:1604.07869 [hep-ph]].
  %%CITATION = doi:10.1007/JHEP09(2016)004;%%
  %20 citations counted in INSPIRE as of 19 Dec 2017

%\cite{vanNeerven:2000uj}
\bibitem{vanNeerven:2000uj}
  W.~L.~van Neerven and A.~Vogt,
  %``NNLO evolution of deep inelastic structure functions: The Singlet case,''
  Nucl.\ Phys.\ B {\bf 588} (2000) 345
  doi:10.1016/S0550-3213(00)00480-6
  [hep-ph/0006154].
  %%CITATION = doi:10.1016/S0550-3213(00)00480-6;%%
  %159 citations counted in INSPIRE as of 19 Dec 2017

%\cite{Bozzi:2005wk}
\bibitem{Bozzi:2005wk}
  G.~Bozzi, S.~Catani, D.~de Florian and M.~Grazzini,
  %``Transverse-momentum resummation and the spectrum of the Higgs boson at the LHC,''
  Nucl.\ Phys.\ B {\bf 737} (2006) 73
  doi:10.1016/j.nuclphysb.2005.12.022
  [hep-ph/0508068].
  %%CITATION = doi:10.1016/j.nuclphysb.2005.12.022;%%
  %362 citations counted in INSPIRE as of 20 Dec 2017

%\cite{Meng:1995yn}
\bibitem{Meng:1995yn}
  R.~Meng, F.~I.~Olness and D.~E.~Soper,
  %``Semiinclusive deeply inelastic scattering at small q(T),''
  Phys.\ Rev.\ D {\bf 54} (1996) 1919
  doi:10.1103/PhysRevD.54.1919
  [hep-ph/9511311].
  %%CITATION = doi:10.1103/PhysRevD.54.1919;%%
  %49 citations counted in INSPIRE as of 21 Dec 2017

%\cite{Collins:2016hqq}
\bibitem{Collins:2016hqq}
  J.~Collins, L.~Gamberg, A.~Prokudin, T.~C.~Rogers, N.~Sato and B.~Wang,
  %``Relating Transverse Momentum Dependent and Collinear Factorization Theorems in a Generalized Formalism,''
  Phys.\ Rev.\ D {\bf 94} (2016) no.3,  034014
  doi:10.1103/PhysRevD.94.034014
  [arXiv:1605.00671 [hep-ph]].
  %%CITATION = doi:10.1103/PhysRevD.94.034014;%%
  %14 citations counted in INSPIRE as of 21 Dec 2017

\end{thebibliography}

\end{document}
